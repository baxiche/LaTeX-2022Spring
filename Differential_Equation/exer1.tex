\documentclass[12pt,a4paper]{article}
%設定頁面
\textwidth 170mm 
\usepackage{geometry}
\geometry{left=2cm=right,top=3cm=bottom}

%Typesetting

%about font shape
\usepackage{fontspec}
%\setmainfont{P22MayflowerPro}
\newcommand{\bt}[1]{%
    \begin{center}{\Large\textbf{#1}}\end{center}}
\newcommand{\bfit}[1]{%
    \textbf{\textit{#1}}}
\newcommand{\bfup}[1]{%
    \textbf{\textup{#1}}}
\newcommand{\mbf}[1]{%
    \textbf{\mathbf{#1}}}
\usepackage{type1cm}

\usepackage{xeCJK} 
\setCJKmainfont{STLibianTC-Regular}
\newCJKfontfamily\Kai{DFKaiShu-SB-Estd-BF} 
\newCJKfontfamily\lie{STLibianTC-Regular}
\XeTeXlinebreaklocale "zh"   
\XeTeXlinebreakskip = 0pt plus 1pt %設定中文
%\renewcommand\contentsname{目錄}

\usepackage[shortlabels]{enumitem}
%\usepackage{enumerate}

%about section
%\renewcommand{\thesection}{\Roman{section}.} 
%\renewcommand{\thesubsection}{\Roman{subsection}.} 
\setcounter{section}{1}
\setcounter{subsection}{1}

%color
\usepackage[dvipsnames]{xcolor}
\definecolor{yel}{RGB}{240,200,90}
\definecolor{3b1bblue}{RGB}{106,177,198}
\definecolor{3b1bred}{RGB}{220,100,90}
\definecolor{Green}{RGB}{50,180,30}
%\pagecolor{backgroud}
%\color{black}

%縮排
\usepackage{indentfirst}
\parindent=0pt 

%頁首、尾文字
\usepackage{fancyhdr}
\pagestyle{fancy}
\chead{}
\lhead{Differential Equation (NTNU MATH, Spring 2022)}
\rhead{Homework \# 1}
\lfoot{}
\cfoot{\thepage}
\rfoot{}
\newcommand{\fancy}[2]{%
    \lfoot{LECTURE #1. #2}\rhead{Lec #1}}
\renewcommand{\headrulewidth}{0.6pt} %上線寬
%\renewcommand{\footrulewidth}{0.4pt} %下線寬
%\renewcommand{\abstractname}{Executive Summary}
%\renewcommand{\chaptername}{}

% Math and some symbol
\usepackage{amsmath, amsthm, xfrac, amssymb}
\DeclareMathOperator{\Area}{Area}
\DeclareMathOperator{\grad}{grad}
\DeclareMathOperator{\curl}{curl}
\DeclareMathOperator{\ord}{ord}
\DeclareMathOperator{\Sp}{span}
\DeclareMathOperator{\rk}{rank}
\DeclareMathOperator{\nul}{nullity}
\DeclareMathOperator{\rref}{rref}
\DeclareMathOperator{\im}{Im}

\newcommand{\re}{\mathbb{R}}
\newcommand{\na}{\mathbb{N}}
\newcommand{\inte}{\mathbb{Z}}
\newcommand{\eucl}{\mathbb{E}}
\newcommand{\ra}{\mathbb{Q}}
\newcommand{\comp}{\mathbb{C}}
\newcommand{\rs}{{\color{red}$\star$}}
\newcommand{\normal}{\trianglelefteq}
\newcommand{\action}{\circlearrowright}



\usepackage{pifont}
\newcommand{\cmark}{\ding{51}}
\newcommand{\xmark}{\ding{55}}
\newcommand{\cirnum}[1]{%
    \ding{\numexpr171+#1\relax}}
\newcommand{\Qed}{%
    \null\nobreak\hfill\ensuremath{\square}}

% defined integral
\newcommand{\Value}[1]{%
    \left.{#1}\phantom{\Big|}\right|}

%very nice
\usepackage{nicematrix}

\usepackage{cancel}
%\usepackage[thicklines]{cancel}
\newcommand{\Ccancel}[2][3b1bblue]{%
    \renewcommand\CancelColor{\color{#1}}\cancel{#2}}
\newcommand{\Dcancel}[2][red!70]{%
    \renewcommand\CancelColor{\color{#1}}\cancel{#2}}
\newcommand{\Ecancel}[2][black]{%
    \renewcommand\CancelColor{\color{#1}}\cancel{#2}}

\newcommand{\red}[1]{%
    {\color{red!70}{#1}}}

\usepackage{mathrsfs} %English Calligraphy
\usepackage{bm}

\usepackage{soul}
\setul{0.5ex}{0.2ex}
\setulcolor{red}
% underline with red color

\usepackage{tikz}
\usepackage{tikz-cd}

\usepackage{mathtools}
%\newcommand{\verteq}{\rotatebox{90}{$\,=$}}
%\newcommand{\equalto}[2]{\underset{\scriptstyle\overset{\mkern4mu\verteq}{#2}}{#1}}
\newcommand{\veq}{\mathrel{\rotatebox{90}{$=$}}}
\newcommand{\vneq}{\mathrel{\rotatebox{90}{$\neq$}}}
\usepackage{extarrows} % arrow with text
\newcommand{\litrom}[1]{%
    \romannumeral#1}
\newcommand{\caprom}[1]{%
    \uppercase\expandafter{\romannumeral#1}}
%羅馬數字

\newtheorem{thm}{Theorem}
\newtheorem*{lemma}{Lemma}
\newtheorem{corollary}{Corollary}

\setcounter{page}{1}
%\pagenumbering{roman}
%\pagestyle{plain}
%設定頁碼

%simply box
\usepackage{varwidth}
\setlength{\fboxrule}{0.6pt}
\newcommand{\varbox}[1]{%
    \fbox{\varwidth{\linewidth}#1\endvarwidth}}

%%%%%%%%%%%%%%%%%%%%%%%%%%%%%%%%%%%%%%%%%%
\newcounter{prop} 
\setcounter{prop}{3}
\renewcommand{\theprop}{\arabic{prop}}
\newcommand{\prop}{%
    \refstepcounter{prop}{\bf{Proposition \theprop.}}}
\newcommand{\cor}{%
    \refstepcounter{prop}{\bfit{Corollary \theprop. }}}

%Throrem in the box
\usepackage[framemethod=TikZ]{mdframed}
%\mdtheorem{boxdef}{Defintion.}
%\newmdtheoremenv{boxx}[lemma]{Theorem}
%\newmdtheoremenv{lemo}{Lemma}
%\newmdenv{}{}

%Just a box
\newcounter{boxthm} 
\setcounter{boxthm}{0}
\renewcommand{\theboxthm}{\arabic{boxthm}}
\newenvironment{boxthm}[2][]{%
    \refstepcounter{boxthm}
    \mdfsetup{%
        innertopmargin=10pt,%
        innerbottommargin=10pt,%
        linecolor=black,%
        linewidth=0.6pt,topline=true,%
        frametitleaboveskip=\dimexpr-\ht\strutbox\relax}
    \begin{mdframed}
        \textbf{\textit{Theorem \theboxthm#1}}\it\relax%
        \label{#2}
    }{\end{mdframed}}

\newcounter{boxdef} 
\setcounter{boxdef}{0}
\renewcommand{\theboxdef}{\arabic{boxdef}}
\newenvironment{boxdef}[2][]{%
    \refstepcounter{boxdef}
    \mdfsetup{%
        innertopmargin=10pt,%
        innerbottommargin=10pt,%
        linecolor=black,%
        linewidth=0.6pt,topline=true,%
        frametitleaboveskip=\dimexpr-\ht\strutbox\relax
    }
    \begin{mdframed}
        \textbf{\textit{Definition \theboxdef#1}}\it\relax%
    \label{#2}
    }{\end{mdframed}}

\newenvironment{boxx}[1][]{%
    \mdfsetup{%
        innertopmargin=10pt,%
        innerbottommargin=10pt,%
        linecolor=black,%
        linewidth=0.6pt,topline=true,%
        frametitleaboveskip=\dimexpr-\ht\strutbox\relax}
    \begin{mdframed}\relax}{\end{mdframed}}
%%%%%%%%%%%%%%%%%%%%%%%%%%%%%%%%%%%%%%%%%%

% Add graph
\usepackage{import}
\usepackage{xifthen}
\usepackage{pdfpages}
\usepackage{transparent}
\newcommand{\incfig}[1]{%
    \def\svgwidth{\columnwidth}
    \import{/Users/baxiche/Documents/LaTeX/Drawing/}{#1.pdf_tex}
}

\title{Solutions of selection problems}
%\author{}
%\date{}
\begin{document}
\fontsize{12pt}{20pt}\selectfont
%\maketitle
%\tableofcontents
Determine the radius of convergence of the given power series.
\begin{enumerate}
    \item[3.] $\displaystyle \sum_{n=0}^{\infty}\frac{x^{2n}}{n!}$. 
    
    \bfit{Solution.} By using the ratio test, we have 
    \[
        \lim_{{n}\to {\infty}} \left|\frac{x^{2(n + 1)}}{(n + 1)!} \frac{n!}{x^{2n}}\right| = \left|x^2\right|\lim_{{n}\to {\infty}} \frac{1}{n + 1} = 0 < 1. 
    \]
    Thus, the series converges absolutely for all $x \in \re$. hence the radius of convergence is $\infty$.
\end{enumerate}
$(11,12)$ Determine the Taylor series about the point $x_0$ for the given function. Also determine the radius of convergence of the series.
\begin{enumerate}
    \item[11.] $\ln x$, $x_0 = 1$. 
    
    \bfit{Solution.} A direct computation gives
    \[
        \frac{1}{1 + x} = \frac{1}{1 - ( - x)} = 1 - x + x^2 - x^3 + \cdots = \sum_{n=0}^{\infty}( - 1)^n x^n.
    \]
    Since $\ln (1 + x) = \int 1 / (1 +x)\ dx$, we have that 
    \[
        \ln (1 + x) = \int_{}^{} \sum_{n=0}^{\infty}( - 1)^n x^n \ dx = \sum_{n=0}^{\infty}  \frac{( - 1)^n}{n + 1} x^{n + 1}.
    \]
    Hence 
    \[
        \ln x = \ln (1 + (x - 1)) = \sum_{n=0}^{\infty} \frac{( - 1)^n}{n + 1}(x - 1)^{n + 1}.
    \]
    \item[12.] $\dfrac{1}{1-x}$, $x_0 = 0$.
    
    \bfit{Solution.} A direct computation gives 
    \[
        \frac{1}{1 - x} = 1 + x + x^2 + \cdots = \sum_{n=0}^{\infty} x^n.
    \]
    \item[14.] Let $\displaystyle y = \sum_{n=0}^{\infty}nx^n$.
    \begin{enumerate}
        \item[a.]Compute $y'$ and write out the first four terms of the series.
        \item[b.]Compute $y^{\prime\prime}$ and write out the first four terms of the series.
    \end{enumerate} 
    \bfit{Solution.} \begin{enumerate}[label=\alph*.,noitemsep,topsep=0pt]
        \item $\displaystyle y' = \sum_{n=1}^{\infty}n ^2 x^{n - 1} = 1 + 4x + 9x ^2 + 16 x ^3 +\cdots $.
        \item $\displaystyle y^{\prime\prime} = \sum_{n=2}^{\infty}n ^2 (n - 1) x^{n - 2} = 4 + 18x + 48x ^2  + 100 x ^3 +\cdots $.                 
    \end{enumerate}
\end{enumerate}
(21) Rewrite as a single power series whose generic term involves $x^n$.
\begin{enumerate}
    \item[21.] $\displaystyle \sum_{n=1}^{\infty}na_nx^{n - 1} + x \sum_{n=0}^{\infty}a_nx^n$. 
    
    \bfit{Solution.} Since
    \[
        \sum_{n = 1}^{\infty} n a_n x^{n - 1} = \sum_{n=0}^{\infty}(n + 1) a_{n + 1}x^n = a_1 +  \sum_{n=0}^{\infty}(n + 1) a_{n + 1}x^n 
    \]
    and 
    \[
        x \sum_{n=0}^{\infty}a_n x^n = \sum_{n=0}^{\infty}a_n x^{n + 1} = \sum_{n = 1}^{\infty}a_{n - 1}x^n.
    \]
    Hence 
    \[
        \sum_{n=1}^{\infty}na_nx^{n - 1} + x \sum_{n=0}^{\infty}a_nx^n = a_1 + \sum_{n = 1}^{\infty} \left((n + 1)a_{n + 1} + a_{n - 1}\right)x^n.
    \]
\end{enumerate}

\begin{enumerate}
    \item[23.] Determine the $a_n$ such that 
    \[
        \sum_{n = 1}^{\infty}n a_n x^{n - 1} + 2\sum_{n=0}^{\infty}a_n x^n = 0.
    \]

    \bfit{Solution.} Since if $\sum a_nx^n$ diverges, then the given equation will never be the case. Hence, we can assume that 
    \[
        \sum_{n=0}^{\infty} a_n x^n = f(x),\ \textup{ for some function } f(x).
    \]
    Then the equation is equivalent to $f'(x) + 2 f(x) = 0.$ Thus, 
    \[
        \dfrac{f'(x)}{f(x)} = - 2 \Rightarrow \ln |f(x)| = - 2x + C \Rightarrow f(x) = Ce^{ - 2x},\textup{\ for some constant }C.
    \]
    Finally, since 
    \[
        \sum_{n=0}^{\infty}a_n x^n =  C e^{ - 2x} = C \cdot \sum_{n=0}^{\infty}\frac{ ( - 2x)^n}{n!},
    \]
    we have that 
    \[
        C = a_0 , \textup{ and }a_n = a_0 \cdot \frac{(-2)^n}{n!},\ \forall n > 0.
    \]
\end{enumerate}



\end{document}
